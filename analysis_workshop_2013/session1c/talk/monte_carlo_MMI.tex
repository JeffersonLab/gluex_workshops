\documentclass[hyperref={pdfpagelabels=false}]{beamer}

\usetheme{AnnArbor}

\newcommand{\bi}{\begin{itemize}}
\newcommand{\ei}{\end{itemize}}
\newcommand{\be}{\begin{enumerate}}
\newcommand{\ee}{\end{enumerate}}
\newcommand{\bc}{\begin{center}}
\newcommand{\ec}{\end{center}}
\newcommand{\bd}{\begin{description}}
\newcommand{\ed}{\end{description}}
\newcommand{\I}{\item}
\newcommand{\f}{\frame}
\newcommand{\ft}{\frametitle}

\title{GlueX Simulation Tools}
\subtitle{GlueX Analysis Workshop}
\author[Mark Ito]{Mark M.\ Ito}
\date{July 10, 2013}
\institute[JLab]{Jefferson Lab}

\begin{document}

\f{\titlepage}

\section{Preliminaries}

\subsection{Terms}

\f{\ft{Defining terms}
\bd
\I[\bf event generation:] specifying initial particles produced
  \bi
  \I initial usually means point of interaction of beam and target particle
  \ei
\I[\bf detector simulation:] conveying particles through the detector
  \bi
  \I tranport
  \I interaction with material
  \I production of secondaries
  \I decay
  \ei
\I[\bf smearing:] introduction of detector resolution at the hit level
  \bi
  \I separate step in GlueX
  \ei
\ed
}

\subsection{HDDM}

\f{\ft{Hall D Data Model (HDDM)}
\bi
\I XML-like
  \bi
  \I binary format
  \I can be rendered as xml: hddm-xml
  \ei
\I every file instance starts with a non-binary template
  \bi
  \I abbreviated schema for the language
  \ei
\I read/write tools can be generated from template
\ei
}

\section{Event Generation}

\f{\ft{Programs}
\bd
\I[\bf particle gun:] generate one-particle events
\I[\bf genr8:] produce X particles in $\gamma p \rightarrow (p\quad {\rm or} n) X$
\I[\bf bggen:] produce events reflecting the total hadronic cross section for photoproduction in Hall D
\ed
}

\subsection{Particle gun}

\f{\ft{Particle gun overview}
\bi
\I useful for developing/testing event reconstruction in a simple context
\I option to HDGEANT, not a stand-alone program
\I can specify particle type, range of directions, range of momenta
\I see description of {\bf\tt hdgeant} below
\ei
}

\subsection{genr8}

\f{\ft{genr8 overview}
\bi
\I useful for calculating acceptance and resolution for signal events
\I generates the reaction $AB \to CD$ via exchange of virtual particle
  \bi
  \I in our context: $\gamma p \rightarrow (p {\rm or} n) X$
  \I cross section assumed $\propto e^{-bt}$ where $t$ is momentum transfer squared, $b$ user-specified slope
  \I final-state particles can decay
    \bi
    \I one decay mode per particle
    \I two-body decay only
    \ei
  \ei
\I configuration via text file and command line arguments
\I output in specialized format
\I output converted with {\tt genr8\_2\_hddm} to HDDM format
\ei
}

\begin{frame}[fragile]

\frametitle{genr8 configuration}

{\footnotesize
\begin{verbatim}
%%%%%%%%%%%%%%%%%%%%%%%%%%%%%%%%%%%%%%%%%%%%%%%%%%%%%%%%%%%%%%%%%%%%%
%
% This file generates event for the following reaction
%
% gamma p -> p X(2000)
%               |
%               |-> b1(1235) pi-
%                    |
%                    |-> omega pi+
%                          |
%                          |-> rho pi0
%                               |   |-> gamma gamma
%                               |
%                               |-> pi+ pi-
%
%
% Feb. 13, 2008  David Lawrence
\end{verbatim}
}
\end{frame}

\begin{frame}[fragile]
\ft{genr8 configuration continued}
{\scriptsize
\begin{verbatim}
% beamp.x beamp.y beamp.z beamMass  
0 0 9 0
% targetp.x targetp.y targetp.z targetMass
0 0 0 0.938
% t-channelSlope
      5.0
% number of particles needed to describe the isobar decay of X
12
%part#  chld1#  chld2#  parent# Id     nchild   mass    width   charge  flag
 0       *      *       *       14       0      0.938   0.0      +1      11  
 1       2      3       *       0        2      2.000   0.100    +1      00
 2       4      5       1       0        2      1.235   0.142    +1      00
 3       *      *       1       9        0      0.140   0.0      -1      11
 4       6      7       2       0        2      0.783   0.009     0      00
 5       *      *       2       8        0      0.140   0.0      +1      11
 6       8      9       4       0        2      0.776   0.150     0      00
 7      10     11       4       7        2      0.135   0.0       0      00
 8       *      *       6       8        0      0.140   0.0      +1      11
 9       *      *       6       9        0      0.140   0.0      -1      11
10       *      *       7       1        0      0.0     0.0       0      11
11       *      *       7       1        0      0.0     0.0       0      11
!EOI
\end{verbatim}
}
\end{frame}

\begin{frame}[fragile]

\ft{genr8 execution}

Usage message from {\tt genr8 -h}
{\small
\begin{verbatim}
genr8 usage: [-A<name>]   < infile 
    -d debug flag
    -n Use a particle name and not its ID number (ascii only) 
    -M<max> Process first max events
    -l<lfevents> Determine the lorentz factor with this many events
    -r<runNo> default runNo is 9000. 
    -P save flag= 11 & 01 events(default saves 11 & 10 events) 
    -A<filename> Save in ascii format. 
    -s<seed> Set random number seed to <seed>. 
             (default is to set using current time + pid) 
    -h Print this help message
\end{verbatim}
}
\end{frame}

\begin{frame}[fragile]

\ft{genr8\_2\_hddm execution}

Usage message from {\tt genr8\_2\_hddm}
{\small
\begin{verbatim}
Usage:
       genr8_2_hddm [options] file.ascii

Convert an ascii file of events generated by genr8 into HDDM
for use as input to hdgeant.

 options:

  -V"x  y  z_min  z_max"    set the vertex for the interaction.
                            (default: x=0 y=0 z_min=65 z_max=65)
  -b"beam_particle_name"    set the beam particle type [gamma].
  -t"target_particle_name"  set the target particle type [proton].
  -h           print this usage statement.
\end{verbatim}
}
\end{frame}

\subsection{bggen}

\f{\ft{bggen overview}
\bi
\I useful for calculating contribution/effect of background events
\I assumes collimated coherent bremsstrahlung beam
\I all hadronic interactions generated in two energy regimes:
  \bi
  \I 0.15 to 3.00~GeV: use measured differential cross sections for 10 dominant processes
  \I above 3.00~GeV use PYTHIA event generator tuned for Hall D conditions
  \ei
\I configuration via text file
\I output in HDDM format
\ei
}

\begin{frame}[fragile]
\ft{bggen configuration}

Text configuration file: {\tt run.ffr}

{\footnotesize
\begin{verbatim}
LIST
TRIG     395000         number of events to simulate
RUNNO    9000           run number of generated events, default is two
C -- writing out events
C        HDDM  simple  ntuple
WROUT      1      1     1   
NPRIEV   100            number of events to print
EPHLIM   0.15 12.       energy range in GeV
RNDMSEQ    0            random number sequence     integer values
EELEC     12.           electron beam energy
EPEAK      9.           coherent peak energy
ZCOLLIM   7600.         distance to the collimator in cm
EPYTHMIN     3.         minimal energy for PYTHIA simulation
STOP
\end{verbatim}
}
\end{frame}

\f{\ft {bggen execution}

\bi
\I input files:
  \bd
  \I[\bf pythia.dat:]contains PYTHIA definitions adjusted for photoproduction (HERMES)
  \I[\bf pythia-geant.map:] mapping of GEANT $\leftrightarrow$ PYTHIA particle codes
  \I[\bf particle.dat:] a list of particle properties used for the low energy mode
  \I[\bf fort.15:] linked to run.ffr--list of commands and definitions in the FFREAD format
  \ed
\I to run, just say {\tt bggen}
\ei
}

\section{detector simulation}
\subsection{hdgeant}

\f{\ft{hdgeant overview}
\bi
\I GEANT 3 implementation
\I hit generation in C
\I geometry definition via HDDS, an XML mark-up language
\I optionally includes electromagnetic background of beam, out-of-time 
\I input from HDDM event file or particle gun
\I output in HDDM format
\ei
}

\begin{frame}[fragile]
\ft{hdgeant configuration}

FFREAD file: control.in

{\scriptsize
\begin{verbatim}
cINFILE 'rhop.hddm'
TRIG 30
cBEAM 12. 9.
OUTFILE 'hdgeant.hddm'
POSTSMEAR 0
DELETEUNSMEARED 0
c MCSMEAROPTS '-t1000 -d0'
c PLOG 1
c TLOG 1
c   particle  momentum  theta  phi  delta_momentum delta_theta delta_phi
KINE   118      1.0       50.   0.      0.              0.        360.
c    vertex_x vertex_y vertex_z
SCAP    0.       0.      65.
c            vertex_extent_r  vertext_extent_z
c TGTWIDTH           0.0              30.0
c    fhalo
HALO  5e-5
BGRATE 1.10
BGGATE -200. 200.
RNDM 121
\end{verbatim}
}
\end{frame}

\begin{frame}[fragile]
\ft{hdgeant configuration continued}
{\scriptsize
\begin{verbatim}
CUTS 1e-4 1e-4 1e-3 1e-3 1e-4
SWIT 0 0 0 0 0 0 0 0 0 0
GELH  1     0.2     1.0     4     0.160
HADR 1
CKOV 1
LABS 1
ABAN 0
DEBU 1 10 1000
NOSECONDARIES 0
TRAJECTORIES 0
c   TMAXFD (REAL) maximum angular deviation due to the magnetic field
c                 permitted in one step (degrees)
c   DEEMAX (REAL) maximum fractional energy loss in one step (0< DEEMAX <=0.1)
c   STEMAX (REAL) maximum step permitted (cm)
c   STMIN  (REAL) minimum value for the maximum step imposed by energy loss,
cAUTO 0
c BFIELDMAP 'Magnets/Solenoid/solenoid_0750_poisson_20091123_03'
c BFIELDTYPE 'Const'
  SAVEHITS  0
  SHOWERSINCOL 0
  DRIFTCLUSTERS 0
\end{verbatim}
}
\end{frame}

\begin{frame}[fragile]
\ft{hdgeant configuration continued further}
{\scriptsize
\begin{verbatim}
c MULS 0 no multiple scattering
c      1 Moliere or Coulomb scattering (default)  
c BREM 0 no bremsstrahlung 
c      1 bremsstrahlung (default)
c COMP 0 no Compton
c      1 Compton scattering (default)
c PAIR 0 no pair production
c      1 pair production (default)
c LOSS 0 (controls energy losses) no energy loss
c      1  delta-rays are produced above the threshold
c      2  no delta-rays are produced. Complete fluctuations are calculated .
c DCAY 0 no decay in flight
c      1 decay in flight with generation of secondaries (default)
c      2 decay in flight without generation of secondaries
c DRAY 0 no delta ray production
c      1 delta ray production with generation of secondaries (default)
c      2 delta ray production without generation of secondaries
END
\end{verbatim}
}
Plus additional standard GEANT cards
\end{frame}

\f{\ft{hdgeant execution}
\bi
\I reads control.in from current working directory
\I to run, just say {\tt hdgeant}
\ei
}

\section{addition of detector resolution}
\subsection{mcsmear}
\f{\ft{mcsmear overview}
\bi
\I hits from hdgeant are perfect, e. g., exact deposited energy
\I new information generated and added to the event with appropriate detector resolution introduced
\I original perfect energy saved or not
\ei
}

\begin{frame}[fragile]
\ft{mcsmear configuration}

Configuration done with command line options

{\scriptsize
\begin{verbatim}
Usage:
     mcsmear [options] file.hddm
  options:
    -ofname  Write output to a file named "fname" (default auto-generate name)
    -N       Add random background hits to CDC and FDC (default is not to add)
    -s       Don't smear real hits (see -B for BCAL, default is to smear)
    -i       Ignore random number seeds found in input HDDM file
    -r"s1 s2 s3" Set initial random number seeds
    -u#      Sigma CDC anode drift time in ns (def:0ns)
    -y       Do NOT apply drift distance dependence error to
             CDC (default is to apply)
    -Y       Apply constant sigma smearing for FDC drift time. 
             Default is to use a drift-distance dependent parameterization.
    -t#      CDC time window for background hits in ns (def:0ns)
    -U#      Sigma FDC anode drift time in ns (def:0ns)
    -C#      Sigma FDC cathode strips in microns (def:0ns)
    -t#      FDC time window for background hits in ns (def:0ns)
    -e       hdgeant was run with LOSS=0 so scale the FDC cathode
             pedestal noise (def:false)
    -d       Drop truth hits (default: keep truth hits)
\end{verbatim}
}
\end{frame}

\begin{frame}[fragile]
\ft{mcsmear configuration continued}
{\scriptsize
\begin{verbatim}
    -p#      FCAL photo-statistics smearing factor in GeV^3/2 (def:0)
    -b#      FCAL single block threshold in MeV (def:0)
    -B       Don't process BCAL hits at all (def. process)
    -Rthresh BCAL TDC threshold (def. 44.7 mV)
    -Wthresh BCAL ADC threshold (def. 4 mV)
    -Wsigma  BCAL fADC time resolution (def. 1.1547 ns)
    -F       Don't smear BCAL energy (def. smear)
    -G       Don't smear BCAL times (def. smear)
    -H       Don't add BCAL dark hits (def. add)
    -K       Don't apply BCAL sampling fluctuations (def. apply)
    -L       Don't apply BCAL sampling floor term (def. apply)
    -M       Don't apply BCAL Poisson statistics (def. apply)
    -Q       Don't apply BCAL time jitter (def. apply)
    -I       Don't apply discrim. thresh. to BCAL hits (def. cut)
    -J       Create BCAL debug hists (only use with a few events!)
    -D       Don't use the deprecated BCAL smearing scheme (def. use deprecated)
    -f#      TOF sigma in psec (def: 100)
    -h       Print this usage statement.
\end{verbatim}
}
\end{frame}

\section{Exercises/Summary}

\f{\ft{Exercises}
\bi
\I Generate events with genr8 (simple isobar model, specific channel)
\I Generate events with bggen (hadronic background)
\I Simulate events with hdgeant
\I Smear events with mcsmear
\ei
}

\end{document}
