\section{Preliminaries}

\ft{Defining terms}

event generation: specifying initial particles produced
  initial usually means point of interaction of beam and target particle
detector simulation: conveying particles through the detector
  tranport
  interaction with material
    production of secondaries
  decay
    production of decay products
smearing: introduction of detector resolution at the hit level
  separate step 

\section{Event Generation}

\ft{Programs}

particle gun: generate one-particle events
genr8: produce X particles in $\gamma p \rightarrow (p {\rm or} n) X$
bggen: produce events reflecting the total hadronic cross section for photoproduction in Hall D

\ft{particle gun overview}

\I useful for developing/testing event reconstruction in a simple context
\I option to HDGEANT, not a stand-alone program
\I can specify particle type, range of directions, range of momenta
\I see description of HDGEANT below

\subsection{genr8}

\ft{genr8 overview}

\I useful for calculating acceptance and resolution for signal events
\I generates the reaction $AB \to CD$ via exchange of virtual particle
  \I in our context: $\gamma p \rightarrow (p {\rm or} n) X$
  \I cross section assumed $\propto e^{-bt}$ where $t$ is momentum transfer squared, $b$ user-specified slope
  \I final-state particles can decay, one decay mode per particle???
  \I decay model???
\I configuration via text file and command line arguments
\I output in specialized format
\I output converted with genr8\_2\_hddm to HDDM format, z-spread added???

\ft{genr8 configuration}

\begin{verbatim}
%%%%%%%%%%%%%%%%%%%%%%%%%%%%%%%%%%%%%%%%%%%%%%%%%%%%%%%%%%%%%%%%%%%%%
%
% This file generates event for the following reaction
%
% gamma p -> p X(2000)
%               |
%               |-> b1(1235) pi-
%                    |
%                    |-> omega pi+
%                          |
%                          |-> rho pi0
%                               |   |-> gamma gamma
%                               |
%                               |-> pi+ pi-
%
%
% Note that in the real world, omegas don't decay into rhos
% but it is implemeted that way here due to the limitations
% of the isobar model.
%
% Feb. 13, 2008  David Lawrence
%%%%%%%%%%%%%%%%% Start Input Values %%%%%%%%%%%%%%%%%%%%
% beamp.x beamp.y beamp.z beamMass  
0 0 9 0
% targetp.x targetp.y targetp.z targetMass
0 0 0 0.938
% t-channelSlope
      5.0
% number of particles needed to describe the isobar decay of X
12
%   
% particle# 0&1 are always the X&Y 
%part#  chld1#  chld2#  parent# Id     nchild   mass    width   charge  flag
% baryon (Y) decay
 0       *      *       *       14       0      0.938   0.0      +1      11  
% meson (X) decay
 1       2      3       *       0        2      2.000   0.100    +1      00
 2       4      5       1       0        2      1.235   0.142    +1      00
 3       *      *       1       9        0      0.140   0.0      -1      11
 4       6      7       2       0        2      0.783   0.009     0      00
 5       *      *       2       8        0      0.140   0.0      +1      11
 6       8      9       4       0        2      0.776   0.150     0      00
 7      10     11       4       7        2      0.135   0.0       0      00
 8       *      *       6       8        0      0.140   0.0      +1      11
 9       *      *       6       9        0      0.140   0.0      -1      11
10       *      *       7       1        0      0.0     0.0       0      11
11       *      *       7       1        0      0.0     0.0       0      11
!EOI
%%%%%%%%%%%%%%%%%%%%% End Input Values %%%%%%%%%%%%%%%%%%%%%%%
\end{verbatim}

\ft{genr8 execution}

\I usage message

\ft{bggen}

\I useful for calculating contribution/effect of background events
\I assumes??? collimated coherent bremsstrahlung beam
\I all hadronic interactions generated in two energy regimes:
  \I ???~GeV to 3~GeV: use measured differential cross sections for 10 dominant processes
  \I 3~GeV to ???~GeV use PYTHIA event generator tuned for Hall D conditions
\I configuration via text file
\I output in HDDM format

detector simulation
  hdgeant
    configuration
    input
    output
addition of detector resolution
  mcsmear
    input
    output
reconstruction
  jana/sim-recon
    input
    output 
analysis
  root tree
hddm
